\documentclass{article}
\usepackage[utf8]{inputenc}
\usepackage{geometry}
\setlength{\parindent}{2em}
\setlength{\parskip}{1em}
\renewcommand{\baselinestretch}{1.3}

\title{Literary Theory Presentation}
\author{Sheriden Goldie}
\date{August 2019}

\begin{document}

\maketitle
\begin{abstract}
    A presentation on Oscar Wilde's "The Decay of Lying" and "The Critic as Artist" as seen in the Norton Anthology of Theory and Criticism (Second Edition).
    This presentation will focus on the questions of:
    \begin{enumerate}
        \item What are the arguments Wilde makes for the imagination, and for criticism as a form of art?
        \item How does the structure of the two pieces reflect Wilde's aesthetic philosophy?
    \end{enumerate}
    
\end{abstract}

\section*{On Aesthetics and Criticism}
\textbf{Timeline}
\begin{itemize}
    \item Oscar Fingal O'Flahertie Wills Wilde was born in Ireland on October 16 1854. He was born to wealthy parents, Sir William Wilde (Surgeon and philanthropist), and Jane Wilde (Sympathizer poet of Irish Nationalism). They were a very important/wealthy family in Dublin. 
\item He was very well educated. He had a natural gift for learning, and did very well in his studies. Though he was also noted for his flamboyance and opulence in decorating his university rooms (at his father’s expense).
\item His first volume of poetry was published in 1881, and sold well but to moderate critical acclaim. 
\item He was made famous somewhat by a Gilbert and Sullivan comic opera called “Patience”, which was a satire of the Aesthetic movement – and one of the characters is said to have resembled Wilde. This connection is more retrospective, but certainly to be a very public face of a movement that is being satirised in one of the most accessible forms did add to Wilde’s social clout. 
\item 1882 - He lectured on Aestheticism in America, his own school of philosophy in terms of art and criticism.
\item 1884 - He married Constance Loyd, and they had two children, Cyril and Vyvyan. 
\item 1887 – He became director of a women’s magazine “Woman’s World”, and turned it on its head by writing articles about the inner emotional life of women (gasp, treating them as complex thoughtful beings).
\item 1890 The Picture of Dorian Grey is published.
\item 1895 - his two better known and received plays were performed. “An Ideal husband”, and “The Importance of Being Earnest”. 
\item 1895 - is also when his life began to unravel. His lover’s (Lord Alfred Douglas  - Bosie) father was the Marquess of Queensberry, and he had taken issue with Wilde “stealing” his son away from his studies – and sent an inflammatory calling card which called Wilde a “posing sodomite”. Wilde actually took this man to court for libel (defamation), and lost. Then he was charged for unnatural practices (homosexuality) and he was at first acquitted, but was then convicted on appeal. He spent two years in prison doing hard labour, and returned to society in 1897 a damaged and sickly man.
\item 1897 - On his release Constance offered reunion if he would go “straight”. Instead Wilde and Bosie reunited, but only for a short while. Wilde lived around Europe calling on the few remaining friends he had.  His wife refused to see him again.
\item 1898 – Published Ballad of Reading Goal. Constance died.
\item 1900 – He dies in Paris from meningitis. It is rumoured that even on his deathbed he was drinking champagne and absinthe.
\end{itemize}
 
\textbf{Discussion}
In the two pieces we have read for class today the first thing that strikes us as unique is the form. These are both dialogues between gentlemen. “The Decay of Lying” is between Cyril and Vivian – the names of Wilde’s children interestingly. “The Critic as Artist” sets an Earnest and a Gilbert in conversation. 

We should note that these are upperclass gentlemen, well educated, and socially “acceptable”. A note on Wilde’s elitism later. 
Wilde was a dramatist, a playwright, and he made a good living from his plays for a time. In setting up the text this way he seems very aware of the need to entertain and engage with counterpoints to his thesis, his argument. The dialogue enables him engages with this very explicitly, as in “The Critic as Artist” on page 796-797

\emph{“Earnest: But, my dear fellow – excuse me for interrupting you – you seem to me to be allowing your passion for criticism to lead you a great deal too far. For, after all, even you must admit that it is much more difficult to do a thing than to talk about it.
Gilbert: More difficult to do a thing than to talk about it? Not at all. That is a gross public error.“ (797)
Wilde is putting forward the act of criticism, to be most valuable as literature and art, is a deeply personal endeavour. That it is through the personal journey and experience one has with art that is the most valuable – and being able to put that experience into words beautifully and elegantly is the highest form of artistic expression.} 

His character Earnest counters these ideas

\emph{“Ernest: Because the best that he (the critic) can give us will be but an echo of rich music, a dim shadow of clear-outlined form…But surely, if this new world has been made by the spirit and touch of a great artist, it will be a thing so complete and perfect that there will be nothing left for the critic to do…
Gilbert: But, surely, Criticism is itself an art. And just as artistic creation implies the working of the critical faculty, and, indeed, without it cannot be said to exist at all, so Criticism is really creative in the highest sense of the word. Criticism is, in fact, both creative and independent. 
Ernest: Independent? 
Gilbert: Yes; independent. Criticism is no more to be judged by any low standard of imitation or resemblance than is the work of poet or sculptor. The critic occupies the same relation to the work of art that he criticises as the artist does to the visible world of form and colour, or the unseen world of passion and of thought.” (798)}

Wilde was working counter to Matthew Arnold – who is mentioned in one of the footnotes. Arnold was a “Sage Writer” a type of writer who instructed the reader on contemporary social issues – even with a slightly finger wagging tone.

He considered the most important criteria used to judge the value of a poem were "high truth" and "high seriousness". By this standard, Chaucer's Canterbury Tales did not merit Arnold's approval. Further, Arnold thought the works that had been proven to possess both "high truth" and "high seriousness", such as those of Shakespeare and Milton, could be used as a basis of comparison to determine the merit of other works of poetry. He also sought for literary criticism to remain disinterested, and said that the appreciation should be of "the object as in itself it really is."

Wilde shifts this –

\emph{“Earnest: But is such work as you have talked about really criticism?
Gilbert: It is the highest criticism, for it criticises not merely the individual work of art, but Beauty itself, and fills with wonder a form which the artist may have left void, or not understood, or understood completely. 
Earnest: The highest criticism, then, is more creative than creation, and the primary aim of the critic is to see the object as in itself it is not.
Gilbert: Yes, that is my theory. To the critic the work of art is simply a suggestion for a new work of his own, that need not necessarily bear any resemblance to the thing it criticises.” (802)}

Wilde is arguing that criticism is a purely subjective form. Unlike Arnold who seemed to believe that one can and should be instructed on how to read a thing/interpret art, that there should be one core meaning. 

\emph{“Gilbert: … Criticism’s most perfect form, which is in its essence purely subjective, and seeks to reveal its own secret and not the secret of another. For the highest Criticsm deals with art not as expressive, but as impressive, purely.”
Art for Wilde is entirely bound up in the impression it makes on a person, and the critics role is to experience that impression absolutely, and allow their own creative work to flow from that, beginning this eternal relation between the personal and art. 
“Gilbert: Yes, from the soul. That is what the highest criticism really is, the record of one's own soul. It is more fascinating than history, as it is concerned simply with oneself. It is more delightful than philosophy, as its subject is concrete and not abstract, real and not vague. It is the only civilised form of autobiography, as it deals not with the events, but with the thoughts of one's life; not with life's physical accidents of deed or circumstance, but with the spiritual moods and imaginative passions of the mind.” (799)}

Wilde was interested in letting souls speak to each other through art. To quote Flannery O’Connor -“I write because I don't know what I think until I read what I say.” In this way for us to engage with each other through art, and creative critical expression is the most authentic ways for our souls to speak to each other – or so Wilde seems to be saying. 

\emph{“… and it is for this very reason that the criticism which I have quoted is criticism of the highest kind. It treats the work of art simply as a starting-point for a new creation.” (801)}

\emph{ “…for the meaning of any beautiful created thing is, at least, as much in the soul of him who looks at it, as it was in his soul who wrought it. Nay, it is rather the beholder who lends to the beautiful thing its myriad meanings, and makes it marvellous for us, and sets it in some new relation to the age, so that it becomes a vital portion of our lives, and a symbol of what we pray for, or perhaps of what, having prayed for, we fear that we may receive.” (801)}

For Wilde beauty reveals everything, because it expresses nothing. The critic must use the art, the object, to be more creative – to show the meaning as well as deepening the mystery/appeal of the art by taking the art form it’s original form into literature. Into language the ultimate tool of art. 

\emph{“Gilbert: … And here, Ernest, this strange thing happens. The critic will indeed be an interpreter, but he will not be an interpreter in the sense of one who simply repeats in another form a message that has been put into his lips to say. For, just as it is only by contact with the art of foreign nations that the art of a country gains that individual and separate life that we call nationality, so, by curious inversion, it is only by intensifying his own personality that the critic can interpret the personality and work of others, and the more strongly this personality enters into the interpretation the more real the interpretation becomes, the more satisfying, the more convincing, and the more true. 
Ernest: I would have said that personality would have been a disturbing element. 
Gilbert: No; it is an element of revelation. If you wish to understand others you must intensify your own individualism.” (805)}

Wilde’s intense reification of the individual comes at a time of equally intense modernisation. We can read it as reflexive, reactive against the booming industry that is sucking up human endeavour to create uniform output in the name of progress. Wilde is rather classist in his views as well, in our extract we end with this:

\emph{“Gilbert:…With us, Thought is degraded by its constant association with practice. Who that moves in the stress and turmoil of actual existence, noisy politician, or brawling social reformer, or poor narrow-minded priest blinded by the sufferings of that unimportant section of the community among whom he has cast his lot, can seriously claim to be able to form a disinterested intellectual judgment about any one thing? Each of the professions means a prejudice. The necessity for a career forces every one to take sides. We live in the age of the overworked, and the under- educated; the age in which people are so industrious that they become absolutely stupid. And, harsh though it may sound, I cannot help saying that such people deserve their doom. The sure way of knowing nothing about life is to try to make oneself useful.” (807)}



\end{document}
